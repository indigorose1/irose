%%%%%%%%%%%%%%%%%%%%%%%%%%%%%%%%%%%%%%%%%
% Short Sectioned Assignment
% LaTeX Template
% Version 1.0 (5/5/12)
%
% This template has been downloaded from:
% http://www.LaTeXTemplates.com
%
% Original author:
% Frits Wenneker (http://www.howtotex.com)
%
% License:
% CC BY-NC-SA 3.0 (http://creativecommons.org/licenses/by-nc-sa/3.0/)
%
%%%%%%%%%%%%%%%%%%%%%%%%%%%%%%%%%%%%%%%%%

%----------------------------------------------------------------------------------------
%	PACKAGES AND OTHER DOCUMENT CONFIGURATIONS
%----------------------------------------------------------------------------------------

\documentclass[paper=a4, fontsize=11pt]{scrartcl} % A4 paper and 11pt font size

\usepackage[T1]{fontenc} % Use 8-bit encoding that has 256 glyphs
\usepackage{fourier} % Use the Adobe Utopia font for the document - comment this line to return to the LaTeX default
\usepackage[english]{babel} % English language/hyphenation
\usepackage{amsmath,amsfonts,amsthm,amssymb} % Math packages

\usepackage{lipsum} % Used for inserting dummy 'Lorem ipsum' text into the template

\usepackage{sectsty} % Allows customizing section commands
\allsectionsfont{\centering \normalfont\scshape} % Make all sections centered, the default font and small caps

\usepackage{fancyhdr} % Custom headers and footers
\pagestyle{fancyplain} % Makes all pages in the document conform to the custom headers and footers
\fancyhead{} % No page header - if you want one, create it in the same way as the footers below
\fancyfoot[L]{} % Empty left footer
\fancyfoot[C]{} % Empty center footer
\fancyfoot[R]{\thepage} % Page numbering for right footer
\renewcommand{\headrulewidth}{0pt} % Remove header underlines
\renewcommand{\footrulewidth}{0pt} % Remove footer underlines
\setlength{\headheight}{13.6pt} % Customize the height of the header
\numberwithin{equation}{section} % Number equations within sections (i.e. 1.1, 1.2, 2.1, 2.2 instead of 1, 2, 3, 4)
\numberwithin{figure}{section} % Number figures within sections (i.e. 1.1, 1.2, 2.1, 2.2 instead of 1, 2, 3, 4)
\numberwithin{table}{section} % Number tables within sections (i.e. 1.1, 1.2, 2.1, 2.2 instead of 1, 2, 3, 4)

\setlength\parindent{0pt} % Removes all indentation from paragraphs - comment this line for an assignment with lots of text

%----------------------------------------------------------------------------------------
%	TITLE SECTION
%----------------------------------------------------------------------------------------

\newcommand{\horrule}[1]{\rule{\linewidth}{#1}} % Create horizontal rule command with 1 argument of height
\newcommand{\argmax}{\arg\!\max}
\title{	
\normalfont \normalsize 
\textsc{Johns Hopkins University} \\ [25pt] % Your university, school and/or department name(s)
\horrule{0.5pt} \\[0.4cm] % Thin top horizontal rule
\huge Statistical Connectomics - Homework 3 \\ % The assignment title
\horrule{2pt} \\[0.5cm] % Thick bottom horizontal rule
}

\author{Sandra I. G\'{o}mez R., Indigo V. L. Rose} % Your name

\date{\normalsize\today} % Today's date or a custom date

\begin{document}

\maketitle % Print the title

%----------------------------------------------------------------------------------------
%	PROBLEM 1
%----------------------------------------------------------------------------------------

\section{Modeling $\mathcal{Z}$}
%\lipsum[2] % Dummy text
Prompt: For the Bock et al. (2011) Connectomics paper, propose a model for  \textit{z} , where \textit{z}  accounts for the directional selectivity of the neurons in the sample space. \\

Solution: \\

(1) Defining the sample space
\\
\\
First, let's define the other components of   $\Xi$ , the sample space. It includes all the neurons in the volume being studied. We have to account for the connectivity of the neurons($\mathcal{X}$ , all the possible undirected graphs that can be constructed with the given number of neurons), their type ($\mathcal{Y}$, whether they are excitatory or inhibitory) and their responsiveness to stimuli ($\mathcal{Z}$, directional selectivity).\\ 

\[\Xi_n =\mathcal{X} \times \mathcal{Y} \times \mathcal{Z}\]
\[\Xi_n = (0,1)^{n\times n} \times \{0,1\}^n \times \mathcal{Z} \]

(2) Defining $\mathcal{Z}$
\\
\\ 
V1 neurons represent directional selectivity, which ranges from 0 to $\pi$. However, the orientations they encode can't be etirely continuous because that would necesitate an infinate number of neurons. So we need to chop up this continuum. Cool. There are many ways to do this. If we want to be super exhaustive, we can make $\mathcal{Z}=[180]^n$, because that's a good estimation of the maximum cognitive sensitivity for the orientation of line segments. But if we are to assume the population coding hypothesis for orientation selectivity, then we can get away with many fewer. A good estimation would be $\mathcal{Z}=[18]^n$, because many neurons are broadly tuned to a maximum output for around 10 degrees. This could even be too many, as the same thing can be accomplished with fewer groups of orientation selectivity (although a higher sensitivity to the firing rate is required, but that's another story), e.g. Bock et al. (2011) used 8. However, to be the most biologically accurate, which might be a good thing when modeling the brain, we propose using the 18 group model:
\[\mathcal{Z}=[18]^n\]

\section{Structure of $\vec{\rho} \ $ and $\ \vec{\beta}$}

In other SBMs, we've made $\vec{\rho} \in \Delta_k$ and $\vec{\beta} \in (0,1)^{k \times k}$. This comes from the most basic definition of SBM, where a scalar quantity (k, number of blocks) and two data structures are used to define the model. One of the data structures, $\vec{\beta}$,  is a matrix that gives the probability that two vertices of different type are connected.  The second data structure, $\ \vec{\rho}$, gives the group index of each node, so it 'tags' each node.

\section{Special Notes}
Modeling  $\mathcal{Z}$ using  stochastic block model (SBM) approach may not be the best approach in this instance. In a SBM, each node (neuron) is assigned to one of many \textit{k} blocks based on the nodes to which they are connected and the connectivity pattern between the nodes. This is not the case of the neurons in the optical cortex, since they are not clustered based on their orientation preference. They simply take the input stimuli, fire based on their preference and send their signal to the closest interneuron in the second cortical layer. Some of them connect to each other, but not based on directional selectivity, so applying a SBM would group neurons in \textit{K} blocks that may not be biologically accurate.

\end{document}